\documentclass[11pt]{beamer}
\usetheme{Warsaw}
\usepackage[utf8]{inputenc}
\usepackage[english]{babel}
\usepackage{amsmath}
\usepackage{amsfonts}
\usepackage{amssymb}

\usepackage[linewidth=1pt]{mdframed}

%expectations
\newcommand{\expect}{\mathbb{E}}

\AtBeginSection[]{
  \begin{frame}
  \vfill
  \centering
  \begin{beamercolorbox}[sep=8pt,center,shadow=true,rounded=true]{title}
    \usebeamerfont{title}\insertsectionhead\par%
  \end{beamercolorbox}
  \vfill
  \end{frame}
}


\begin{document}
%%%%%%%%%%%%%%%%%%%%%%%%%%%%%%%%%%%%%%%%%%%%%%%%%%%%%%%%
\begin{frame}
  \frametitle{}
  \begin{center}
    \textbf{\large MATH 4281 Risk Theory--Ruin and Credibility}\\
    \vspace{1cm}
    {\large  Module 2 Bonus: Some other applications of Ruin Theory} \\
    \vspace{1cm}
    {\large  March 2, 2021}
    \end{center}
    \vspace{1cm}
\end{frame}
%%%%%%%%%%%%%%%%%%%%%%%%%%%%%%%%%%%%%%%%%%%%%%%%%%%%%%%%
\begin{frame}
\tableofcontents
\end{frame}
%%%%%%%%%%%%%%%%%%%%%%%%%%%%%%%%%%%%%%%%%%%%%%%%%%%%%%%%
\section{Review of the It\^{o} calculus}
\begin{frame}{The building blocks}

Define the Wiener process $W_t$ by:

\begin{itemize}
\item $W_0= 0$

\item $W_t$ is continuous 

\item $W_t$ has independent increments

\item $W_{t}-W_{s}\sim {\mathcal {N}}(0,t-s)$

\end{itemize}

\vfill

Recall we can recover this as the limit of a random walk as the number of steps goes to infinity. 

\end{frame}
%%%%%%%%%%%%%%%%%%%%%%%%%%%%%%%%%%%%%%%%%%%%%%%%%%%%%%%%
\begin{frame}{It\^{o} Integrals}

\begin{itemize}

\item We can then define integrals with respect to $W_t$.
\vfill
\item Assume $f_t$ is \emph{adapted to} $W_t$. Fancy way of saying it shares the probability space. 
\vfill
\item Take a partition of $[0,t]$ into $n$ intervals denoted $\pi_n$ and:

$$ \int _{0}^{t}f_t\,dW=\lim _{n\rightarrow \infty }\sum _{[t_{i-1},t_{i}]\in \pi _{n}}f_{t_{i-1}}(W_{t_{i}}-W_{t_{i-1}}) $$

\end{itemize}

\end{frame}
%%%%%%%%%%%%%%%%%%%%%%%%%%%%%%%%%%%%%%%%%%%%%%%%%%%%%%%%
\begin{frame}{It\^{o} Processes}

\begin{itemize}
\item We can then construct \emph{It\^{o} Processes}: 

$$ X_{t}=X_{0}+ \int _{0}^{t}\mu _{s}\,ds + \int _{0}^{t}\sigma _{s}\,dB_{s} $$

\vfill

\item Or in differential form:

$$ dX_{t}=X_{0}+ \mu _{t}dt + \sigma _{t}dB_{t} $$
\end{itemize}

\end{frame}
%%%%%%%%%%%%%%%%%%%%%%%%%%%%%%%%%%%%%%%%%%%%%%%%%%%%%%%%
\begin{frame}{Doing Calculus with Random Variables}

\begin{itemize}

\item Given an It\^{o} Process- how does a function of it behave (Real world example: a derivative price as a function of a random stock)
\vfill
\item Assume $X_t$ satisfies $ dX_{t}=X_{0}+ \mu _{t}dt + \sigma _{t}dB_{t} $
\vfill
\item Assume that $f(t,X)$ is $C^2(\mathbb{R})$ then:

    $$df(X_t)=\left({\frac {\partial f}{\partial t}}+\mu _{t}{\frac {\partial f}{\partial x}}+{\frac {\sigma _{t}^{2}}{2}}{\frac {\partial ^{2}f}{\partial x^{2}}}\right)dt+\sigma _{t}{\frac {\partial f}{\partial x}}\,dB_{t}$$
\vfill

\item This is the famous It\^{o}'s lemma

\end{itemize}

\end{frame}
%%%%%%%%%%%%%%%%%%%%%%%%%%%%%%%%%%%%%%%%%%%%%%%%%%%%%%%%
\begin{frame}{Generators} 

\begin{itemize}

\item Let $\tau$ be a stopping time (recall from our lectures)
\vfill
\item We have a nice result called Dynkin's formula:

$$\mathbf{E} [f(X_{\tau})] = f(X_0) + \mathbf{E}^{x} \left[ \int_{0}^{\tau} A f (X_{s}) \, \mathrm{d} s \right] $$

Where $A$ is the \emph{generator of} $X_t$
\vfill
\item In previous slide this would mean: 

$$A = {\frac {\partial }{\partial t}}+\mu _{t}{\frac {\partial }{\partial x}}+{\frac {\sigma _{t}^{2}}{2}}{\frac {\partial ^{2}}{\partial x^{2}}} $$
\vfill
\item There is a deep link between the algebra of differential operators and stochastic processes. Hence why PDEs are common in finance and insurance. 

\end{itemize}

\end{frame}
%%%%%%%%%%%%%%%%%%%%%%%%%%%%%%%%%%%%%%%%%%%%%%%%%%%%%%%%
\begin{frame}{Examples}

\begin{itemize}

\item Brownian Motion with drift: 

$$ dX_t = \mu dt + \sigma dW_t $$

\item Geometric Brownian Motion: 

$$ dX_t = \mu X_t dt + \sigma X_t dW_t $$

\item Ornstein–Uhlenbeck (Mean Reversion) process:

$$ dX_t = -\theta X_t dt + \sigma X_t dW_t $$

\end{itemize}

\end{frame}
%%%%%%%%%%%%%%%%%%%%%%%%%%%%%%%%%%%%%%%%%%%%%%%%%%%%%%%%
\section{Computing Ruin Probabilities}
\begin{frame}{ A simple example } 

\begin{itemize}

\item For general processes computing Ruin Probabilities involves the solution to a very complex Partial Integro-Differential Equation (PIDE). But sometimes we can get lucky.

\vfill

\item Consider a simple BM with drift. We start at $A_0$ and have and \emph{additive} wealth dynamic:

\begin{equation}\label{additive}
dA_t = A_0 + \mu dt + \sigma dW_t 
\end{equation}

\vfill 
\item We want to apply out optimal stopping theorem technique we learned so we will construct a martingale by guessing: 

\begin{equation}\label{martingale}
M_t = e^{ - (\alpha A_t - A_0) }
\end{equation}

\end{itemize}

\end{frame}
%%%%%%%%%%%%%%%%%%%%%%%%%%%%%%%%%%%%%%%%%%%%%%%%%%%%%%%%
\begin{frame}{ A simple example } 
\begin{enumerate}

\item[1] First we need to guarantee (\ref{martingale}) is a martingale. Applying It\^{o}'s Lemma: 

\begin{align*}
dM_t = \left( -\alpha \mu M_t + \alpha^2\frac{ \sigma^2}{2} M_t \right)dt + \left( -\alpha \sigma M_t \right)dW_t
\end{align*}

\vfill

\item[2] Setting the drift equal to zero gives $\alpha=\frac{2\mu}{\sigma^2}$. This makes $M_t$ a \textit{local martingale} but given the integrability of $M_t$ it is a martingale as well. 

\end{enumerate}
\end{frame}
%%%%%%%%%%%%%%%%%%%%%%%%%%%%%%%%%%%%%%%%%%%%%%%%%%%%%%%%
\begin{frame}{ A simple example } 
\begin{enumerate}

\item[3] Define our stopping time as $\tau = \inf\left\lbrace t| M_t>a \text{ or } M_t<b \right\rbrace$. 

\vfill

\item[4] Applying the optimal stopping theorem:

\begin{align*}
E[M_0] &= E[M_\tau] \\
1 &= e^{-\alpha (b-A_0)} P( M_\tau = b ) + e^{-\alpha (b-A_0)} (1-P( M_\tau = b )) \\
P( M_\tau = b ) &= \frac{ e^{ -\alpha A_0 } -  e^{ -\alpha a } }{e^{ -\alpha b } -  e^{ -\alpha a }}
\end{align*}

\vfill

\item[5] Send $b \rightarrow \infty$ and $a \rightarrow 0$ and we have the ruin probability is $1-e^{-\alpha A_0}$ and the survival probability its $e^{-\alpha A_0}$.

\end{enumerate}
\end{frame}
%%%%%%%%%%%%%%%%%%%%%%%%%%%%%%%%%%%%%%%%%%%%%%%%%%%%%%%%
\begin{frame}{Remarks}

\begin{itemize}

\item So if we maximize $\alpha$ we minimize ruin!
 

\vfill

\item Interestingly this can be extended to other processes (using a more complex proof). 

\vfill

\item That is minimizing $\frac{\mu_t}{\sigma_t}$ where $\mu_t$ and $\sigma_t$ are the drift and diffusion parts of the generator $A$ minimizes ruin. A very useful result!

\end{itemize}

\end{frame}
%%%%%%%%%%%%%%%%%%%%%%%%%%%%%%%%%%%%%%%%%%%%%%%%%%%%%%%%
\begin{frame}{Remarks}

\begin{itemize}

\item Notice this is a similar result to the Lundberg inequality. In fact it is only the discontinuity of the CL process that prevents an exact match. 

\vfill

\item Not that surprising: If we have probability distributions that are exponentially bounded (Markov inequality) then for some limit we should see exponentially behaved probabilities. 

\vfill

\item What if we don't have this...?

\end{itemize}

\end{frame}
%%%%%%%%%%%%%%%%%%%%%%%%%%%%%%%%%%%%%%%%%%%%%%%%%%%%%%%%
\section{Investing your insurance float}
\begin{frame}{Do what?!}

\begin{itemize}

\item Often an insurance company will invest it's premiums. 

\vfill

\item Famous example of this is Warren Buffet. Buffet actually accessed a smaller cost of capital than other investors by investing his insurance companies surplus or "float".

\vfill

\item  There is also something called "convergence capital" where low bond yields are forcing reinsurers to invest their premiums in hedge funds (personally I think this is a bad idea...but no one asked me).

\end{itemize}

\end{frame}
%%%%%%%%%%%%%%%%%%%%%%%%%%%%%%%%%%%%%%%%%%%%%%%%%%%%%%%%
\begin{frame}

\begin{itemize}

\item Consider an investor who invests in a risky asset $S_t$ described by a GBM and a bank account $B_t$ with interest rate $r$ i.e:

\begin{equation*}\label{S/B eq}
dS_t = \mu S_t dt + \sigma S_t dW^{(1)}_t
\quad\text{and}\quad 
dB_t = r B_t dt
\end{equation*}

\item Their wealth $X$ evolves according to the SDE:  
\vspace{0.1cm}
\begin{align*}
X_t &= B_t + \gamma S_t  \\
dX_t &= r B_t dt + \gamma S_t ( \mu dt + \sigma dW^{(1)}_t)  \\
dX_t &= \underbrace{ B_t }_{(1-f)X_t} rdt + \underbrace{\gamma S_t}_{ fX } ( \mu dt + \sigma dW^{(1)}_t )  \\
dX_t &=  X_t [f(\mu-r) + r ]dt + [ X_t f\sigma ]dW^{(1)}_t\label{X dyn}
\end{align*}

\vfill

\item $f$ is the (potentially dynamic) fraction of total wealth $X_t$ invested in the risky asset.

\end{itemize}

\end{frame}
%%%%%%%%%%%%%%%%%%%%%%%%%%%%%%%%%%%%%%%%%%%%%%%%%%%%%%%%
\begin{frame}{Take our model}

\begin{itemize}

\item Take the CL model of net claims we studied in class:

$$ Y_t = ct - \sum_{i=1}^{N(t)}X_i $$
\vfill

\item Take $a=c - \lambda E[X]$ and $b^2=\lambda E[X^2]$
\vfill

\item We can approximate $Y_t$ by a BM with drift (more reasonably for some times scales and parameters than others):

$$ Y_t \approx adt + b dW^{(2)}_t $$ 

\end{itemize}

\end{frame}
%%%%%%%%%%%%%%%%%%%%%%%%%%%%%%%%%%%%%%%%%%%%%%%%%%%%%%%%
\begin{frame}{Putting it together}

\begin{itemize}

\item If we add the net insurance claims our model for the insurance company now becomes:

$$ dX_t =  X_t [f(\mu-r) + r ]dt + [ X_t f\sigma ]dW^{(1)}_t +  adt + b dW^{(2)}_t $$

\item From the generator of this process we can now get:

\begin{align*}
\mu_t &= X_t [f(\mu-r) + r ] + a \\
\sigma_t &= [ X_t f\sigma ]^2 + b^2 + 2\rho b [ X_t f\sigma ]
\end{align*}

\end{itemize}

\end{frame}
%%%%%%%%%%%%%%%%%%%%%%%%%%%%%%%%%%%%%%%%%%%%%%%%%%%%%%%%
\begin{frame}{Putting it together}
Finding the $f$ that maximizes $\frac{\mu}{\sigma^2}$ gives:

\vfill
{\small
$$ f_\psi = \frac{1}{\mu-r}\left[ \sqrt{ \left( rx+a - \frac{\rho b (\mu -r)}{\sigma} \right)^2 + (1-\rho^2)b^2\frac{(\mu-r)^2}{\sigma^2} } - (rx + a) \right] $$
}

\vfill
\end{frame}

\section{What is Optimal: Kelly vs Ruin vs ?}
\begin{frame}{A few Questions}

\begin{itemize}

\item Say $b=\rho=0$ and $a=-c$ i.e. some consumption an investor may need to satisfy. You can show that:

$$f_{\psi}^* = \begin{cases} 
      \frac{2|rx-c|}{x\sigma^2}\frac{ 1 }{f_{g}^{*}} & rx-c<0 \\
      0 & rx-c \geq 0
   \end{cases}$$
   
\vfill
\item Where $f_{g}^{*}$ is the "Kelly" or growth optimal fraction. 
   
\vfill
\item So a Ruin theoretic approach will be much much more conservative...which is correct?

\end{itemize}

\end{frame}


\end{document}