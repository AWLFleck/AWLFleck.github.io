\documentclass[11pt]{beamer}
\usetheme{Warsaw}
\usepackage[utf8]{inputenc}
\usepackage[english]{babel}
\usepackage{amsmath}
\usepackage{amsfonts}
\usepackage{amssymb}

%expectations
\newcommand{\expect}{\mathbb{E}}

\AtBeginSection[]{
  \begin{frame}
  \vfill
  \centering
  \begin{beamercolorbox}[sep=8pt,center,shadow=true,rounded=true]{title}
    \usebeamerfont{title}\insertsectionhead\par%
  \end{beamercolorbox}
  \vfill
  \end{frame}
}


\begin{document}
%%%%%%%%%%%%%%%%%%%%%%%%%%%%%%%%%%%%%%%%%%%%%%%%%%%%%%%%
\begin{frame}
  \frametitle{}
  \begin{center}
    \textbf{\large MATH 4281 Risk Theory--Ruin and Credibility}\\
    \vspace{1cm}
    {\large  Intro to the course and start of Module 1} \\
    \vspace{1cm}
    {\large  January 12, 2021}
    \end{center}
    \vspace{1cm}
\end{frame}
%%%%%%%%%%%%%%%%%%%%%%%%%%%%%%%%%%%%%%%%%%%%%%%%%%%%%%%%
\begin{frame}
\tableofcontents
\end{frame}
%%%%%%%%%%%%%%%%%%%%%%%%%%%%%%%%%%%%%%%%%%%%%%%%%%%%%%%%
\section{Introduction to this course}
\begin{frame}{The story so far...}

For those who took Math 4280 you essentially studied the following problem: 
\vfill
\begin{center}
\fbox{\textbf{How to compute $\rho[L]$?}}
\end{center}
\vfill
Where:
\begin{itemize}

\item $\rho$ is some (potentially) coherent risk measure e.g. ES/TCE/CVaR, etc...

\item $L$ is some random variable representing a loss.

\end{itemize}

\end{frame}
%%%%%%%%%%%%%%%%%%%%%%%%%%%%%%%%%%%%%%%%%%%%%%%%%%%%%%%%
\begin{frame}{Some Questions}

Q1: What do you do when $L$ is equal to a sum of smaller RVs?

\vfill

Q2: How do you introduce \textbf{time} to this model?

\vfill

Q3: How do I estimate the parameters of the model for $L$...if I don't have a nice heterogeneous sample?

\end{frame}
%%%%%%%%%%%%%%%%%%%%%%%%%%%%%%%%%%%%%%%%%%%%%%%%%%%%%%%%

\begin{frame}{Some Questions}

Q1: What do you do when $L$ is equal to a sum of smaller RVs? \\
\color{red}$\Rightarrow$  Module 1: Aggregate Loss Models
\vfill
\color{black}Q2: How do you introduce \textbf{time} to this model? \\
\color{red}$\Rightarrow$  Module 2: Ruin Theory
\vfill


\color{black} Q3: How do I estimate the parameters of the model for $L$...if I don't have a nice heterogeneous sample? \\
\color{red}$\Rightarrow$  Module 3: Credibility 

\end{frame}

%%%%%%%%%%%%%%%%%%%%%%%%%%%%%%%%%%%%%%%%%%%%%%%%%%%%%%%%
\section{Start of  Module 1: Intro to Aggregate Loss Models}
\begin{frame}{Models for aggregate losses}
  A portfolio of insurance contracts or an insurance contract will potentially experience a sequence of losses:
  \vfill
  $$X_1, X_2, X_3, \ldots$$
  \vfill
  We are interested in the aggregate sum $S$ of these losses over a certain period of time.

\end{frame}
%%%%%%%%%%%%%%%%%%%%%%%%%%%%%%%%%%%%%%%%%%%%%%%%%%%%%%%%
\begin{frame}{Assumptions going forward}
\begin{itemize}
    
    \item How do losses relate to each other? 
    \begin{itemize}
    \item[$\hookrightarrow$] Assume independent $X_i$'s.
    \end{itemize}
    
\vfill    
    
    \item When do these losses occur?
    \begin{itemize}
    \item[$\hookrightarrow$] Assume no time value of money i.e. short term models 
    \end{itemize}
  
\vfill  
  
  \item How many losses will occur?
    \begin{itemize}
    \item[$\hookrightarrow$] if deterministic ($n$) $\longrightarrow$ individual risk model
    \item[$\hookrightarrow$] if random ($N$) $\longrightarrow$ collective risk model
    \end{itemize}



\end{itemize}
\end{frame}
%%%%%%%%%%%%%%%%%%%%%%%%%%%%%%%%%%%%%%%%%%%%%%%%%%%%%%%%
\begin{frame}{Definition: The Individual Risk Model}

The Individual Risk Model\footnote{Refer to Def. 9.2 in the loss model textbook}

 $$S=X_{1}+\cdots +X_{n}=\sum_{i=1}^{n}X_{i},$$
 
 
  \begin{itemize}
  \item The random variables $X_{i}$, $i=1,2,...,n$, are assumed to be independent 
  \item BUT they are not assumed to be identically distributed.
  \item Typically the $X_i$'s have mass at $0$ (representing no loss/payment).
\end{itemize}

\end{frame}


%%%%%%%%%%%%%%%%%%%%%%%%%%%%%%%%%%%%%%%%%%%%%%%%%%%%%%%%
\begin{frame}{Definition: The Collective Risk Model}
In the Collective Risk Model, aggregate losses become
$$S=X_{1}+\ldots +X_{\alert{N}}=\sum_{i=1}^{\alert{N}}X_{i}.$$
This is a random sum. We make the following assumptions:
\begin{itemize}
\item $N$ is the number of claims
\item $X_i$ is the amount of the $i$th claim
\item the $X_i$'s are \alert{iid} with
\begin{itemize}
\item CDF $F(x)$
\item P(D/M)F $f(x)$
\item Moments exist and are finite!
\end{itemize}
\item the $X_i$'s and $N$ are mutually independent
\end{itemize}
\end{frame}
%%%%%%%%%%%%%%%%%%%%%%%%%%%%%%%%%%%%%%%%%%%%%%%%%%%%%%%%
\begin{frame}{How to compute $S$?}
We will study methods to get probabilities about $S$:
\vfill
\begin{enumerate}

  \item If possible we will get the true distribution of $S$ via:
  
    \begin{itemize}
    \item Convolutions
    \item Method of generating functions
    \end{itemize}
    \vfill
  \item Otherwise we will approximate with the help of the moments of $S$
  
\end{enumerate}

\end{frame}

%%%%%%%%%%%%%%%%%%%%%%%%%%%%%%%%%%%%%%%%%%%%%%%%%%%%%%%%
\begin{frame}{Examples of The IRM vs CRM}

\vspace{- 1 cm}

A group life insurance contract where each employee has a different age, gender,
and death benefit\footnote{ex 9.2 of the loss models book}.
 
\vfill

\end{frame}
%%%%%%%%%%%%%%%%%%%%%%%%%%%%%%%%%%%%%%%%%%%%%%%%%%%%%%%%
\begin{frame}{Examples of The IRM vs CRM}
\vspace{- 1 cm}
A reinsurance contract that pays when the annual total medical malpractice costs at a certain hospital exceeds a given amount.

\vfill

\end{frame}
%%%%%%%%%%%%%%%%%%%%%%%%%%%%%%%%%%%%%%%%%%%%%%%%%%%%%%%%
\begin{frame}{Examples of The IRM vs CRM}
\vspace{- 1 cm}
A dental policy on an individual pays for at most two checkups per year per family
member. A single contract covers any size family at the same price.

\vfill

\end{frame}
%%%%%%%%%%%%%%%%%%%%%%%%%%%%%%%%%%%%%%%%%%%%%%%%%%%%%%%%
\begin{frame}{An example for the collective Risk model}
\vspace{- 1 cm}
An insurable event has a 10\% probability of occurring and when it occurs results in a loss of \$5,000. Market research has indicated that consumers will pay at most \$550 to purchase insurance against this event. How many policies must a company sell in order to have a 95\% chance of making money (ignoring expenses)?\footnote{ex 9.1 of the loss models book}
\vfill
\end{frame}


%%%%%%%%%%%%%%%%%%%%%%%%%%%%%%%%%%%%%%%%%%%%%%%%%%%%%%%%
\begin{frame}{An example for the collective Risk model}

\end{frame}
%%%%%%%%%%%%%%%%%%%%%%%%%%%%%%%%%%%%%%%%%%%%%%%%%%%%%%%%
\begin{frame}{An example for the collective Risk model}

\end{frame}
%%%%%%%%%%%%%%%%%%%%%%%%%%%%%%%%%%%%%%%%%%%%%%%%%%%%%%%%
\section{Generating Functions and Convolutions}
\subsection{Generating Functions}
%%%%%%%%%%%%%%%%%%%%%%%%%%%%%%%%%%%%%%%%%%%%%%%%%%%%%%%%
\begin{frame}{Probability Generating functions}
  
\begin{definition}[PGF]
Given a discrete RV $X$. We define the Probability Generating Function(PGF) $p_{X}(t)$ as:

$$p_{X}\left( t\right)=E\left[ t^{X}\right]  $$ 

\end{definition}

i.e:

$$p_{X}\left( t\right)=Pr[X=x_0]t^{x_0}+\Pr[X=x_1] t^{x_1}+ \Pr[X=x_2]  t^{x_2} + \ldots$$

\end{frame}
%%%%%%%%%%%%%%%%%%%%%%%%%%%%%%%%%%%%%%%%%%%%%%%%%%%%%%%%
\begin{frame}{Properties}
  \begin{itemize}

  \item There is a 1-1 relation between a distribution and its PGF.
  \item If $X$ is an integer-valued random variable, then the PGF is
    $$p_{X}\left( t\right)=E\left[ t^{X}\right] =\sum_{n=0}^\infty\Pr[X=n] t^n,$$
    which is in fact the Taylor series of $p_{X}\left( t\right)$:

$$\Pr[X=n]=\frac{\frac{d^n}{dt^n}p_{X}\left( t\right)|_{t=0}}{n!}$$.

  \item If $X_i, i=1,\cdots,n$ are independent, then: $$\alert{p_{X_1+\cdots+X_n}\left( t\right)}=\alert{p_{X_1}\left( t\right)}\alert{p_{X_2}\left( t\right)}\cdots \alert{p_{X_n}\left( t\right)}$$
  \end{itemize}
\end{frame}
%%%%%%%%%%%%%%%%%%%%%%%%%%%%%%%%%%%%%%%%%%%%%%%%%%%%%%%%
\begin{frame}{Moment Generating functions}

\begin{definition}[MGF]

For a continous random variable $X$ we define the Moment Generating Function (MGF) as: 
$$M_{X}\left( t\right)=E\left( e^{tX}\right)$$

\end{definition}

\end{frame}
%%%%%%%%%%%%%%%%%%%%%%%%%%%%%%%%%%%%%%%%%%%%%%%%%%%%%%%%
\begin{frame}{Properties}
  \begin{itemize}

  \item There is a 1-1 relation between a distribution and its MGF.

  \item Taylor Expansion:

$$ M_{X}\left( t\right)=1+E[X] t+ E[X^2]  \frac{t^2}{2} +E[X^3] \frac{t^3}{6} + \ldots + E[X^k] \frac{t^k}{k!}+\ldots $$  
  

 and thus
      $$E[X^k]=\left.\dfrac{d^{k}}{dt^{k}}m_{X}\left( t\right) \right|_{t=0}$$

  \item If $X_i, i=1,\cdots,n$ are independent then: $$\alert{M_{X_1+\cdots+X_n}\left( t\right)}=\alert{M_{X_1}\left( t\right)}\alert{M_{X_2}\left( t\right)}\cdots \alert{M_{X_n}\left( t\right)}$$
  \end{itemize}
\end{frame}
%%%%%%%%%%%%%%%%%%%%%%%%%%%%%%%%%%%%%%%%%%%%%%%%%%%%%%%%
\begin{frame}{Why do we care?}

\begin{itemize}
  \item As we said there is a 1-1 relation between a distribution and its MGF or PGF.
  \vfill
  \item Sometimes, $m_S(t)$ or $p_S(t)$ can be recognised: this is the case for infinitely divisible distributions (Normal, Poisson, Inverse Gaussian, \ldots) and certain other distributions (Binomial, Negative binomial)
  \vfill
  \item Otherwise, $m_S(t)$ or $p_S(t)$ can be expanded \linebreak numerically to get moments and/or probabilities
  \end{itemize}

\end{frame}
%%%%%%%%%%%%%%%%%%%%%%%%%%%%%%%%%%%%%%%%%%%%%%%%%%%%%%%%
\begin{frame}
\vspace{-2.5 cm}
  \textbf{Example}
  Consider a portfolio of 10 contracts. The losses $X_i$'s for these contracts are i.i.d. Poisson RVs with parameter 100.  Determine the distribution of $S$.
\end{frame}
%%%%%%%%%%%%%%%%%%%%%%%%%%%%%%%%%%%%%%%%%%%%%%%%%%%%%%%%
\begin{frame}

\end{frame}
%%%%%%%%%%%%%%%%%%%%%%%%%%%%%%%%%%%%%%%%%%%%%%%%%%%%%%%%
\begin{frame}

\end{frame}
%%%%%%%%%%%%%%%%%%%%%%%%%%%%%%%%%%%%%%%%%%%%%%%%%%%%%%%%
\begin{frame}
\vspace{-2.5 cm}
\textbf{Example} Consider three independent RVs $X_1$, $X_2$, $X_3$. For $i=1,2,3$, $X_i$ has an exponential distribution and $E[X_i]=1/i$.
  Derive the PDF of $S=X_1+X_2+X_3$ by recognition of the MGF of $S$.
\end{frame}
%%%%%%%%%%%%%%%%%%%%%%%%%%%%%%%%%%%%%%%%%%%%%%%%%%%%%%%%
\begin{frame}

\end{frame}
%%%%%%%%%%%%%%%%%%%%%%%%%%%%%%%%%%%%%%%%%%%%%%%%%%%%%%%%
\begin{frame}

\end{frame}
%%%%%%%%%%%%%%%%%%%%%%%%%%%%%%%%%%%%%%%%%%%%%%%%%%%%%%%%
\subsection{Convolutions}
\begin{frame}{Convolutions}
\begin{itemize}

\item The operation of computing the distribution of the sum of two independent random variables is called a \alert{convolution}. It is denoted by:


$$F_{X+Y}=F_X*F_Y$$

\vfill

\item The result can then be convoluted with the distribution of another random variable:
$$F_{X+Y+Z}=F_Z*F_{X+Y}$$
\vfill
\item And so on...(as we will see for $n$-fold convolutions)


\end{itemize}
\end{frame}
%%%%%%%%%%%%%%%%%%%%%%%%%%%%%%%%%%%%%%%%%%%%%%%%%%%%%%%%
\begin{frame}{Formulas}

Continuous case:
   \begin{itemize}
    \item CDF: $F_{X+Y}\left( s\right) =\int_{-\infty }^{\infty}F_{Y}\left( s-x\right) f_{X}\left( x\right) dx$
    \item PDF: $f_{X+Y}\left( s\right) =\int_{-\infty }^{\infty}f_{Y}\left( s-x\right) f_{X}\left( x\right) dx$
  \end{itemize}
  
\vfill  

Discrete case:

\begin{itemize}
    \item CDF: $F_{X+Y}\left( s\right) =\sum_{x}F_{Y}\left( s-x\right)f_{X}\left( x\right) $
    \item PMF: $f_{X+Y}\left( s\right) =\sum_{x}f_{Y}\left(s-x\right) f_{X}\left( x\right) $
    \end{itemize}

\end{frame}
%%%%%%%%%%%%%%%%%%%%%%%%%%%%%%%%%%%%%%%%%%%%%%%%%%%%%%%%
\begin{frame}{$n$-fold convolutions}
  For i.i.d. continuous random variables $X_i$ with a common CDF $F_X (x)$, the $n$-fold convolution of $F_X (x)$ is denoted by $F^{*n}_X (x)$: 
  
\begin{eqnarray*}
    F^{*k}_{X}\left( x\right) &=& \int^{\infty}_{-\infty} F^{*(k-1)}_{X} \left(x-y\right) f_X (y) d y\\
    &=& \int^{x}_{0} F^{*(k-1)}_{X} \left(x-y\right) f_X (y) d y \quad \mbox{if positive support}
    \end{eqnarray*}
     for $k = 1, 2, \ldots$ where:  
     
 
 
  $$F^{*0}_{X}\left( x\right) =
    \left\{
    \begin{aligned}
    & 0 , & x < 0 \\
    & 1 , & x \geq 0
    \end{aligned}
    \right.$$
  
\end{frame}
%%%%%%%%%%%%%%%%%%%%%%%%%%%%%%%%%%%%%%%%%%%%%%%%%%%%%%%%
\begin{frame}{$n$-fold convolutions (Cont.)}
Continuous case PDF for $k = 1, 2, \ldots$ :
 
\begin{eqnarray*}
    f^{*k}_{X}\left( x\right) &=& \int^{\infty}_{-\infty} f^{*(k-1)}_{X} \left(x-y\right) f_X (y) d y\\
    &=& \int^{x}_{0} f^{*(k-1)}_{X} \left(x-y\right) f_X (y) d y \quad \mbox{if positive support}
    \end{eqnarray*}
    
 
\end{frame}
%%%%%%%%%%%%%%%%%%%%%%%%%%%%%%%%%%%%%%%%%%%%%%%%%%%%%%%%
\begin{frame}{$n$-fold convolutions (Cont.)}
Discrete case:
      \begin{itemize}
    \item CDF:
    \begin{eqnarray*}
    F^{*k}_{X}\left( x\right) = \sum^{x}_{y = 0} F^{*(k-1)}_{X} \left(x-y\right) f_X (y) \ \mbox{for} \ x=0,1,\ldots, \ k = 2, 3, \ldots
    \end{eqnarray*}
    \item PMF:
    \begin{eqnarray*}
    f^{*k}_{X}\left( x\right) = \sum^{x}_{y = 0} f^{*(k-1)}_{X} \left(x-y\right) f_X (y) \ \mbox{for} \ x=0,1,\ldots, \ k = 2, 3, \ldots
    \end{eqnarray*}
  \end{itemize}
\end{frame}
%%%%%%%%%%%%%%%%%%%%%%%%%%%%%%%%%%%%%%%%%%%%%%%%%%%%%%%%
\begin{frame}{Why?}
\vspace{-3 cm}
  \textbf{Exercise:} Show the convolution gives the distribution for the sum. 
  
  
\end{frame}
%%%%%%%%%%%%%%%%%%%%%%%%%%%%%%%%%%%%%%%%%%%%%%%%%%%%%%%%
\begin{frame}

\end{frame}
%%%%%%%%%%%%%%%%%%%%%%%%%%%%%%%%%%%%%%%%%%%%%%%%%%%%%%%%
\end{document}